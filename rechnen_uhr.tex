% Autor: Marvin Ferber
\documentclass[14pt,ngerman]{extarticle}
\usepackage{amsmath, amssymb, datetime, calculator, enumitem, tikz}
\usepackage[a4paper,margin=1in,top=2.5cm]{geometry}
\usepackage{pgf, fp}
\usepackage[ngerman]{babel}
\usepackage[utf8]{inputenc}
\usepackage[T1]{fontenc}
\usepackage{lmodern}
\usepackage{tikz}
\usepackage{xparse}
\usepackage{xspace}
%%%%%%%%%%%%%%%%%%%%%%%%%%%%%%%%%%%%%%%%%%%%%%%%%
\newcommand{\leadingzero}[1]{\ifnum #1<10 0#1\else#1\fi}    
%%%%%%%%%%%%%%%%%%%%%%%%%%%%%%%%%%%%%%%%%%%%%%%%%
% Zufallsgenerator initialisieren
\def\TotalSecondsMinute{}%
\def\TotalSeconds{}%
\MULTIPLY{\currentminute}{60}{\TotalSecondsMinute}%
\ADD{\TotalSecondsMinute}{\currentsecond}{\TotalSeconds}%
\pgfmathsetseed{\TotalSeconds}%
% Globale Variablen
\pgfmathsetmacro{\Aglob}{random(2,5)}
\pgfmathsetmacro{\Bglob}{random(2,5)}
\pgfmathsetmacro{\Cglob}{random(2,5)}
\pgfmathsetmacro{\Dglob}{random(1,7)}

\newcommand\clock[2]{%
    \begin{tikzpicture}[line cap=round,line width=3pt]
        \filldraw [fill=lightgray!30] (0,0) circle (2cm);
        \foreach \angle / \label in
            {0/3, 30/2, 60/1, 90/12, 120/11, 150/10, 180/9,
                210/8, 240/7, 270/6, 300/5, 330/4}
            {
                \draw[line width=1pt] (\angle:1.8cm) -- (\angle:2cm);
                \draw (\angle:1.4cm) node{\textsf{\label}};
            }
        \foreach \angle in {0,90,180,270}
        \draw[line width=2pt] (\angle:1.6cm) -- (\angle:2cm);
        \draw[rotate=90,line width=2pt] (0,0) -- (-#1*30-#2*30/60:0.7cm); % hours
        \draw[rotate=90,line width=1.5pt] (0,0) -- (-#2*6:1cm); % minutes
        %\draw[rotate=90,thin,red] (0,0) -- (-#3*6:1.2cm); % seconds
        \path [fill=red] (0,0) circle (2pt);
        %
    \end{tikzpicture}%
}
\newcommand\clockempty{%
    \begin{tikzpicture}[line cap=round,line width=3pt]
        \filldraw [fill=lightgray!30] (0,0) circle (2cm);
        \foreach \angle / \label in
            {0/3, 30/2, 60/1, 90/12, 120/11, 150/10, 180/9,
                210/8, 240/7, 270/6, 300/5, 330/4}
            {
                \draw[line width=1pt] (\angle:1.8cm) -- (\angle:2cm);
                \draw (\angle:1.4cm) node{\textsf{\label}};
            }
        \foreach \angle in {0,90,180,270}
        \draw[line width=2pt] (\angle:1.6cm) -- (\angle:2cm);
        %\draw[rotate=90,line width=2pt] (0,0) -- (-#1*30-#2*30/60:0.7cm); % hours
        %\draw[rotate=90,line width=1.5pt] (0,0) -- (-#2*6:1cm); % minutes
        %\draw[rotate=90,thin,red] (0,0) -- (-#3*6:1.2cm); % seconds
        \path [fill=red] (0,0) circle (2pt);
        %
    \end{tikzpicture}%
}

%%%%%%%%%%%%%%%%%%%%%%%%%%%%%%%%%%%%%%%%%%%%%%%%%
\newcommand\clockrandomviertel{
    \pgfmathsetmacro{\A}{random(0,11)}
    \pgfmathsetmacro{\B}{random(0,3)}
    %\MULTIPLY{\A}{10}{\A}
    \MULTIPLY{\B}{15}{\B}
    \FPtrunc\A{\A}{0}
    \FPtrunc\B{\B}{0}
    \clock{\A}{\B}
}

\newcommand\clockrandomfuenfer{
    \pgfmathsetmacro{\A}{random(0,11)}
    \pgfmathsetmacro{\B}{random(0,11)}
    %\MULTIPLY{\A}{10}{\A}
    \MULTIPLY{\B}{5}{\B}
    \FPtrunc\A{\A}{0}
    \FPtrunc\B{\B}{0}
    \clock{\A}{\B}
}

\newcommand\timerandomfuenfer{
    \pgfmathsetmacro{\A}{random(0,11)}
    \pgfmathsetmacro{\B}{random(0,11)}
    %\MULTIPLY{\A}{10}{\A}
    \MULTIPLY{\B}{5}{\B}
    \FPtrunc\A{\A}{0}
    \FPtrunc\B{\B}{0}
    \leadingzero{\A}:\leadingzero{\B}
}

\newcommand\timerandomviertel{
    \pgfmathsetmacro{\A}{random(0,11)}
    \pgfmathsetmacro{\B}{random(0,3)}
    %\MULTIPLY{\A}{10}{\A}
    \MULTIPLY{\B}{15}{\B}
    \FPtrunc\A{\A}{0}
    \FPtrunc\B{\B}{0}
    \leadingzero{\A}:\leadingzero{\B}
}

\newcommand\stundenrandom{
    \pgfmathsetmacro{\A}{random(1,4)}
    \FPtrunc\A{\A}{0}
    \A
}

\newcommand\minutenrandom{
    \pgfmathsetmacro{\A}{random(0,11)}
    \MULTIPLY{\A}{5}{\A}
    \FPtrunc\A{\A}{0}
    \A
}

\newcommand\timerandomkuchen{
    %\pgfmathsetmacro{\A}{random(0,11)}
    \pgfmathsetmacro{\B}{random(0,11)}
    %\MULTIPLY{\A}{10}{\A}
    \MULTIPLY{\B}{5}{\B}
    %\FPtrunc\A{\A}{0}
    \FPtrunc\B{\B}{0}
    11:\leadingzero{\B}
}
%%%%%%%%%%%%%%%%%%%%%%%%%%%%%%%%%%%%%%%%%%%%%%%%%
\title{
    \Huge Arbeitsblatt Uhr 
    \vspace{-1cm}
    \date{\today}
    \vspace{-2cm}
}
%%%%%%%%%%%%%%%%%%%%%%%%%%%%%%%%%%%%%%%%%%%%%%%%%
\begin{document}
\begin{center}
    \Huge Arbeitsblatt Uhr 
    \\
    \today
\end{center}        
%\maketitle
\subsection*{Lies ab!}
\begin{tabular}{ c c c }
    \clock{2}{15} & \clockrandomviertel & \clockrandomviertel \\
    02:15 & \verb|__:__| & \verb|__:__|  \\
    14:15 & \verb|__:__| & \verb|__:__|  \\
    viertel 3 & \textunderscore\textunderscore\textunderscore\textunderscore & \textunderscore\textunderscore\textunderscore\textunderscore \\
    & & \\
\end{tabular}
\\
\begin{tabular}{ c c c }
    \clockrandomfuenfer & \clockrandomfuenfer & \clockrandomfuenfer \\
    \verb|__:__| & \verb|__:__| & \verb|__:__|  \\
    \verb|__:__| & \verb|__:__| & \verb|__:__|  \\
    \textunderscore\textunderscore\textunderscore\textunderscore & \textunderscore\textunderscore\textunderscore\textunderscore & \textunderscore\textunderscore\textunderscore\textunderscore \\
    & & \\
\end{tabular}
\\
\begin{tabular}{ c c c }
    \clockrandomfuenfer & \clockrandomfuenfer & \clockrandomfuenfer \\
    \verb|__:__| & \verb|__:__| & \verb|__:__|  \\
    \verb|__:__| & \verb|__:__| & \verb|__:__|  \\
    \textunderscore\textunderscore\textunderscore\textunderscore & \textunderscore\textunderscore\textunderscore\textunderscore & \textunderscore\textunderscore\textunderscore\textunderscore \\
    & & \\
\end{tabular}

\pagebreak
\subsection*{Zeichne ein!}
\begin{tabular}{ c c c }
    \clockempty & \clockempty & \clockempty \\
    \timerandomviertel & \timerandomviertel & \timerandomviertel  \\
    \verb|__:__| & \verb|__:__| & \verb|__:__|  \\
    \textunderscore\textunderscore\textunderscore\textunderscore & \textunderscore\textunderscore\textunderscore\textunderscore & \textunderscore\textunderscore\textunderscore\textunderscore \\
    & & \\
\end{tabular}
\\
\begin{tabular}{ c c c }
    \clockempty & \clockempty & \clockempty \\
    \timerandomfuenfer & \timerandomfuenfer & \timerandomfuenfer  \\
    \verb|__:__| & \verb|__:__| & \verb|__:__|  \\
    \textunderscore\textunderscore\textunderscore\textunderscore & \textunderscore\textunderscore\textunderscore\textunderscore & \textunderscore\textunderscore\textunderscore\textunderscore \\
    & & \\
\end{tabular}

\subsection*{Berechne die Dauer!}
\subsubsection*{a)}
Ein Mann steigt um 14:10 in den Zug. Er fährt\stundenrandom\xspace Stunden und\minutenrandom\xspace Minuten. Wann kommt er an?

\subsubsection*{b)}
Eine Frau bäckt Kuchen. Sie stellt den Kuchen um 10:35 Uhr in den Backofen. Um\timerandomkuchen\xspace Uhr ist der Kuchen fertig. Wie lange hat der Kuchen gebacken? 

\subsubsection*{c)}
Oma Eierschecke besucht ihre Enkel. Sie kommt um 19:45 Uhr am Banhof an. Ihr Zug fuhr\stundenrandom\xspace Stunden und\minutenrandom\xspace Minuten. Wann ist sie mit dem Zug gestartet?
  %
\end{document}
